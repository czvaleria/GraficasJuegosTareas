\documentclass[12pt,letterpaper,fleqn]{article}

\usepackage[utf8]{inputenc}
\usepackage{tikz}
\usepackage[utf8]{inputenc}
\usepackage[T1]{fontenc}
\usepackage{amsmath}
\usepackage{amssymb}
\usepackage{multicol}
\usepackage{graphicx}
\usepackage{mdwlist}
\usepackage{ upgreek }
\usepackage{ stmaryrd }
\usepackage[dvipsnames]{xcolor}
\usepackage[most]{tcolorbox} 
\usepackage{tabu}
\usepackage{mathtools}
\usepackage[top=1in, bottom=1in, left=1in, right=1in]{geometry}
\usepackage{listings}
\usepackage{courier}


\begin{document}

\include{Portada/portada}
    
\begin{center}
    \LARGE{\textbf{Tarea 07}}
\end{center}

\begin{enumerate}
      \item Una gr\'afica $G$ es autocomplementaria si $G \cong \overline{G}$.
        Demuestre que si $G$ es autocomplementaria, entonces $|V|
        \stackrel{4}{\equiv} 0$ o $|V| \stackrel{4}{\equiv} 1$.
        
        $\underline{Respuesta:}$

        Supongamos que $G$ es una gráfica autocomplementaria, es decir aquella gráfica que es isomorfa a su complemento.

        Por definición, el complemento de $G$, tiene los mismos vértices que $G$, pero sus aristas son las que no están en $G$. Es decir, si $(u,v)$ es una arista de $G$, pero no lo es en $\overline{G}$.

        Por definición, si existe un isomorfismo entre $G$ y $\overline{G}$, implica que no solo haya una correspondencia uno a uno entre los vértices de las gráficas, si no también entre sus aristas.
        
        Entonces ambos tienen el mismo número de aristas:
        \begin{center}
            $|E(G)|=|E(\overline{G})|$
        \end{center}

        Ahora como $G$ y $\overline{G}$ son disjuntos y cubren todas las aristas que puede tener la gráfica completa $G$ podemos decir que:
        
        \begin{center}
            $|E(G)|+|E(\overline{G})| = \binom{n}{2}$
            
            $\Rightarrow$ $|E(G)|= \frac{\binom{n}{2}}{2}$ Por definición de gráfica autocomplementaria.
            
            $\Rightarrow |E(G)|=\frac{\frac{n(n-1)}{2}}{2}$
            
            $\Rightarrow |E(G)|=\frac{n(n-1)}{4}$
        \end{center}

        \begin{enumerate}
            \item Supongamos que $|V|\stackrel{4}{\equiv} 0$ es verdadero.
            
            Quiere decir que hay algún $k \in \mathbb{Z}$ para $n=4k$.
            
            Entonces $n$ puede ser $4,8,12,16,...$

            Sustituimos algún valor de $n$ en la operación de $|E(G)|=\frac{n(n-1)}{4}$

            \begin{itemize}
                \item Para $n=4$

                $\frac{n(n-1)}{4} \Rightarrow \frac{4(4-1)}{4}=\frac{4(3)}{4} =\frac{12}{4}=3$

                \item Para $n=8$

                $\frac{n(n-1)}{4} \Rightarrow \frac{8(8-1)}{4}=\frac{8(7)}{4} =\frac{56}{4}=14$

                \item Para $n=12$

                $\frac{n(n-1)}{4} \Rightarrow \frac{12(12-1)}{4}=\frac{12(11)}{4} =\frac{132}{4}=33$

            \end{itemize}

            $\therefore |V|\stackrel{4}{\equiv} 0$ es verdadero $\blacksquare$

            \item Supongamos que $|V|\stackrel{4}{\equiv} 1$ es verdadero.

            Quiere decir que hay algún $k \in \mathbb{Z}$ para $n=4k+1$

            Entonces $n$ puede ser $1,5,9,13,...$

            Sustituimos algún valor de $n$ en la operación de $|E(G)|=\frac{n(n-1)}{4}$

            \begin{itemize}
                \item Para $n=1$

                $\frac{n(n-1)}{4} \Rightarrow \frac{1(1-1)}{4}=\frac{1(0)}{4} =\frac{0}{4}=0$

                \item Para $n=5$

                $\frac{n(n-1)}{4} \Rightarrow \frac{5(5-1)}{4}=\frac{5(4)}{4} =\frac{20}{4}=5$

                \item Para $n=9$

                $\frac{n(n-1)}{4} \Rightarrow \frac{9(9-1)}{4}=\frac{9(8)}{4} =\frac{72}{4}=18$
            \end{itemize}

             $\therefore |V|\stackrel{4}{\equiv} 1$ es verdadero $\blacksquare$
     
        \end{enumerate}        
        
          
  \item Un {\em orden topol\'ogico} de una digr\'afica $D$ es un orden lineal de
    sus v\'ertices tal que para cada flecha $a$ de $D$, la cola de $a$ precede a
    su cabeza en el orden.
    \begin{enumerate}
      \item Demuestre que toda digr\'afica ac\'iclica tiene al menos una fuente
        (v\'ertice de ingrado $0$) y un sumidero (v\'ertice de exgrado $0$).

    \underline{Respuesta:} Sea $D$ una digráfica acíclica. Es decir, no contiene ciclos dirigidos.
    
        Primero, demostraremos que existe al menos una fuente.
        
        Supongamos por contradicción que todos los vértices tienen al menos una arista entrante. Es decir, para todo vértice $v$ existe algún vértice $u$ tal que $(u,v)$.
        
        Consideremos la  trayectoria dirigida de longitud máxima en $D$ $(v_{0}, v_{1}), (v_{1}, v_{2}), \dots, (v_{k-1}, v_{k})$
        
        Como $v_0$ es el primer vértice de la trayectoria y por hipótesis tiene al menos una arista entrante, existe un vértice $u$ tal que también está $(u, v_{0})$. Entonces, podemos extender la trayectoria: $(u, v_{0}), (v_{0}, v_{1}),…, (v_{k-1}, v_{k})$. Esto contradice el hecho de que la trayectoria era de longitud máxima. Por lo tanto, debe existir al menos un vértice sin aristas entrantes.
                
        $\therefore$ Existe una fuente.
        
        Ahora, demostraremos que existe al menos un sumidero.
        
        Supongamos por contradicción que todos los vértices tienen al menos una arista saliente. Es decir, para todo vértice $v$ existe algún vértice $w$ tal que $(v, w)$. 
        
        Tomamos nuevamente una trayectoria dirigida de longitud máxima $(v_0, v_1), (v_1, v_2), \dots, (v_{k-1}, v_k)$. El vértice $v_k$ es el último de la trayectoria. Por hipótesis, tiene al menos una arista saliente, es decir, existe $w$ tal que existe el arco $(v_k, w) $. Entonces podemos extender la trayectoria $(v_0, v_1), (v_1, v_2), \dots, (v_{k-1}, v_k), (v_k, w)$. Esto contradice el hecho de que la trayectoria era de longitud máxima. Por lo tanto, debe existir al menos un vértice sin aristas salientes.
        
        $\therefore$ Existe un sumidero.
        
        Por lo tanto, toda digráfica acíclica tiene al menos una fuente y al menos un sumidero. $\bigstar$
        
      \item Deduzca que una digr\'afica admite un orden topol\'ogico si y s\'olo
        si es ac\'iclica.

        \underline{Respuesta:} Sea $D$ una digráfica.

        $\Longrightarrow )$ Supongamos $D$ tiene un orden topológico en sus vértices. Esto significa que para toda flecha $(v_{i}, v_{j})$, se cumple que $i<j$, es decir, las flechas solo están dirigidas en orden.
        
        Por demostrar:  $D$ es acíclica.
    
        Procedemos por contradicción. Supongamos que $D$ tiene un ciclo $(v_{1}, v_{2}, …, v_{k}, v_{1})$. Como es un ciclo, los vértices se repiten, y en particular $v_{1}$ aparece al inicio y al final. Entonces existe una flecha $(v_{k}, v_{1})$ con la que deberíamos tener $k<1$, sin embargo, no cumple el orden topológico (de hipótesis), ya que no es posible que una flecha este orientada de regreso, pues esta debe de orientarse a un siguiente vértice. Así, no puede haber ciclos en $D$.
    
        $\therefore$ Es acíclica.
    
        $\Longleftarrow) $ Sea $D$ una digráfica acíclica.
        
        Por demostrar: $D$ tiene un orden topológico.
    
        Procedemos por inducción sobre el número de vértices de $D$.
    
        Casos base:
        \begin{itemize}
            \item Caso 1: Sea $n = 1$, $D$ tiene un sólo vértice y ningún arco, entonces el orden se cumple inmediato.
            \item Caso 2: Sea n = 2, entonces tenemos dos casos:
            \begin{itemize}
                \item Si no hay arcos, entonces el orden (o cualquier otro) se cumple inmediato.
                \item Si hay un arco $(v_{1}, v_{2})$, entonces el orden es $v_{1}, v_{2}$ , el cual cumple $i < j$ pues $v_{1}$ debe aparecer antes de $v_{2}$ .       
            \end{itemize}
        \end{itemize}
            
        Hipótesis de Inducción: Supongamos que toda digráfica acíclica con $n = m$ vértices admite un orden topológico.
        
        Paso Inductivo: Sea $D$ una digráfica sin ciclos con $m + 1$ vértices.
        
        Por demostrar: Toda digráfica sin ciclos dirigidos con $n=m + 1$ vértices también tiene un orden topológico.
        
        Como $D$ es aciclica (por hipótesis), entonces debe existir al
        menos un vértice $v$ con $ingrado=0$, es decir, es una fuente. De lo contrario, si todos los vértices tuvieran al menos un arco entrante, podríamos construir un ciclo dirigido siguiendo los arcos entrantes, lo que contradice la hipótesis.
        
        Ahora construyamos una nueva digráfica $D’$. Eliminamos el vértice $v$ y todos sus arcos salientes, con lo que obtenemos a $D’$ , la cual tiene $m$ vértices y es acíclica, pues eliminar vértices no crea ciclos.
        
        Por H.I, $D’$ tiene un orden topológico $(v_{1} , v_{2}, ..., v_{m})$ en los vértices de $D’$ tal que todo arco $(v_{i}, v_{j})$ cumple $i<j$.
        
        Ahora, insertemos el vértice $v$ al inicio del orden. Entonces tenemos $(v, v_{1} , v_{2} , ..., v_{m})$. Como $v$ es una fuente, no hay arcos entrantes a el en $D$. En particular, los únicos posibles arcos que lo involucran son de la forma $(v, v_{i})$ y estos se orientan hacia un siguiente vértice en el orden, pues $v$ aparece antes que cualquier otro vértice.
        
        $\therefore$ $D$ tiene un orden topológico. 

        Por lo tanto, toda digráfica acíclica tiene al menos una fuente y un sumidero. $\bigstar$
        
      \item Exhiba un algoritmo de tiempo a lo m\'as cuadr\'atico para encontrar
        un orden topol\'ogico en una digr\'afica ac\'iclica.
    
    \underline{Respuesta:} Proponemos el siguiente algortimo en el que vamos iterando sobre los vértices: eliminamos los vértices que tienen ingrado $0$ (las fuentes), uno por uno, actualizamos los grados de los demás vértices, y lo agregamos al orden. 
    \\
    
    Input: Una digráfica acíclica G, sus V vértices y A arcos.
    
    Output: Una lista con los vértices en orden topológico.
    
    \\
    \begin{flushleft}
    1. orden $\longleftarrow$ [ ] ; Inicializamos una lista vacía para guardar el orden topológico. \\
    
    2. ingrado $\longleftarrow$ [ ] ; Creamos una lista \textit{ingrado} para guardar cuántas flechas entran a cada vértice, y los inicializamos en $0$. \\
    
    3. \textbf{for each} vértice $v$ en $V(G)$ \textbf{do} \\
    
    4.\hspace{1em} ingrado[$v$] $\longleftarrow$ 0 \\
    
    5. \textbf{for each} arco $(u,v)$ en $G$ \textbf{do} ; Recorremos todos los arcos $(u,v)$ de $G$ y actualizamos los ingrados. \\
    
    6.\hspace{1em} ingrado[$v$] $\longleftarrow$ ingrado[$v$] $+ 1$ \\
    
    7. \textbf{while} exista un vértice $v$ con $ingrado = 0$ \textbf{do} ; Ahora trabajamos con los vértices que tienen $ingrado = 0$ (fuentes). Esto se debe a que en toda digráfica acíclica siempre existe al menos un vértice sin flechas entrantes (una fuente), pues si no existiera tal vértice, podríamos construir un ciclo. \\
    
    8.\hspace{1em} añadir $v$ a [orden] ; Añadimos al vértice a la lista [orden] \\
    
    9.\hspace{1em} eliminar $v$ de $G$ ; Eliminamos al vértice de la gráfica para marcarlo como procesado. \\
    
    10.\hspace{1em} \textbf{for each} vértice $w$ al que $v$ apunta \textbf{do} ; Actualizamos los ingrados disminuyendo su ingrado en 1 para cada vértice al que apunta $v$. \\
    
    11.\hspace{2em} ingrado[$w$] $\longleftarrow$ ingrado[$w$] $- 1$ \\
    
    12. \textbf{if} longitud de [orden] $=$ $|V(G)|$ \textbf{then} ; Verificamos si el orden es válido, es decir, si procesamos todos los vértices. \\
    
    13.\hspace{1em} \textbf{return} [orden] ; El orden es válido, no hay ciclos. \\
    
    14. \textbf{else return} [ ] ; Había un ciclo, pues quedó algún vértice con $ingrado > 0$. \\
    
    15. El proceso termina.
    \end{flushleft}

    Esto se repite $n$ veces, porque hay $n$ vértices, por lo que la parte de verificar si un vértice tiene ingrado cero es $O(n)$ y la parte de eliminar los vértices y actulizar el ingrado es $O(n)$. Por lo tanto, su complejidad es $O(n^2)$. $\bigstar$
        \end{enumerate}

  \item Demuestre que cada uno de ni los siguientes problemas est\'a en la clase
    $NP$ exhibiendo un certificado y un algoritmo de tiempo polinomial para
    verificar el certificado (escriba el algoritmo utilizando pseudo c\'odigo
    como el visto en clase; s\'olo est\'a permitido el uso de las estructuras de
    control {\bf if}, {\bf while} y {\bf for}).   Demuestre que su algoritmo
    usa tiempo polinomial.
    \begin{enumerate}
      \item \textsc{Hamilton Cycle}.
      \underline{Respuesta:} Queremos ver que existe un ciclo que pase por todos los vértices exactamente una vez.

      1. if el |C| no es igual al |V| then
      \\      
      2. \hspace{1em} return False ; Esto porque falta o sobran vértices.
      \\
      3. if hay vértices repetidos en C then
      \\
      4. \hspace{1em} return False; Porque no se puede repetir ciudades en el recorrido.
      \\
      5. for each par de vértices consecutivos en C
      \\
      6.\hspace{1em} if no hay una arista entre ellos en G then 
      \\
      7.\hspace{2em} return False
      \\
      8. if no hay arista entre el último y el primer vértice de C then
      \\
      9. \hspace{1em} return False; Porque no termina donde empieza
      \\
      10. return True. ; Es un ciclo hamiltoniano válido.
      \\
    
    Esto se repite $n$ veces, porque hay $n$ vértices, por lo que la parte de verificar repeticiones es $O(n)$ y la parte de las arsitas que son consecutivas igual es $O(n)$. Por lo tanto, su complejidad es $O(n^2)$. $\bigstar$
    \\

      \item \textsc{Vertex Cover}.
      \underline{Respuesta:} Queremos ver qué existe un conjunto de vértices $S$ de tamaño $\leq k$ que cubra todas las aristas, es decir, que toda arista tiene al menos un extremo en $S$. 

      Input: Una gráfica G con sus V vértices y E aristas, un conjunto S y un entero k. 

        Output: True si $S$ el tamaño es menor a K y False en caso contrario.
        
        1. if el tamaño de $S > k$ then
        \\
        2. \hspace{1em} return False ; S no cumple el límite de tamaño.
        \\
        3. for each arista $(u, v)$ en E
        \\
        4. \hspace{1em} if  u no está en S y v no está S then
        \\
        5. \hspace{2em} return False ; Porque $(u, v)$ no está cubierta por S.
        \\
        6. \hspace{2em} return True 
        \\
        7. return True ; Es que todas las aristas están cubiertas por S.
        \\
        
        Esto se repite $n$ veces, porque hay $n$ vértices. Por lo tanto, su complejidad es $O(n)$. $\bigstar$

        
      \item \textsc{Colouring}.
      \underline{Respuesta:} Proponemos el siguiente algoritmo para ver si es posible colorear el grafo con $\leq k$ colores, sin que vértices adyacentes compartan color. 

    Input: Una gráfica $G$ con sus $V$ vértices y $E$ aristas, una lista $C$ con la asignación de coloraciones y un entero $k$.
    
    Output: True si $C$ es una coloración válida con $\leq k$ colores o False en caso contrario.

1. if el número de colores en $C > k$ then 
\\
2. return False ; Pues se usaron más colores de los permitidos.
\\
3. for each arista $(u, v)$ en E do
\\
4. \hspace{1em} if C[u] == C[v] then
\\
5. \hspace{2em} return False ; Porque hay vecinos con el mismo color. 
\\
6. \hspace{2em} else return True ; La coloración es válida.

El algoritmo recorre cada $n$ arista una  vez, entonces el algoritmo es $O(n)$. $\bigstar$

      \item \textsc{Dominating Set}.
      \underline{Respuesta:} Queremos ver que existe un conjunto $S$ de tamaño$\leq k$ tal que todo vértice fuera del conjunto es adyacente a algún vértice en $S$. $\bigstar$
      
    \end{enumerate}
\end{enumerate}

\subsection*{Puntos extra}
\begin{enumerate}
  \item Demuestre que toda digr\'afica sin lazos admite una
    descomposici\'on en dos digr\'aficas ac\'iclicas, es decir, que
    existen $D_1$ y $D_2$ subdigr\'aficas de $D$, ac\'iclicas y
    tales que $D_1 \cup D_2 = D$ y $A_{D_1} \cap A_{D_2} =
    \varnothing$.

    $\underline{Respuesta:}$

    Sea $D = (V,A)$ una digráfica sin lazos. 

    Usando inducción sobre el número de aristas $m=|E|$.

    $\underline{Caso}$ $\underline{Base}: m=0$

    Si $D$ no tiene asistas, entonces la digrafica $D$ es solo un conjunto de vértices sin adyacencia. En este caso, tanto $D_1$ como $D_2$ son simplemente el mismo grado $D$, que es acíclico por definición.

    $\therefore D_1 = D_2 = D = (V,\varnothing)$. Ambas subgráficas son acíclicas, $A_1\cap A_2=\varnothing=A$. Cumpliéndose así el las descomposiciones en el enunciado.

    $\underline{Hipotesis}$ $\underline{Inductiva}:$

    Es cierto que para toda digráfica sin lazos con $m$ aristas, existe una descomposición en dos gráficas acíclicas.
    
    P.d. Que para una digráfica $D=(V,A)$ con $m+1$ aristas.

    Sea $e=(u,v) \in A$ una arista arbitraria de la digráfica $D$ y la eliminamos. Resultando a una nueva digráfica $D'=(V,A')$, donde $A'=A-\{e\}$

    Entonces $D'$ tiene $m$ aristas y, por hipótesis inductiva, existe una descomposición tal que:

    \begin{itemize}
        \item $D_1'=(V,A_1')$ y $D'_2=(V,A_2')$ son subgráficas acíclicas.
        \item $A_1'\cap A_2'=\varnothing$
        \item $A_1'\cap A_2'=A'$
    \end{itemize} 
    
    Ahora necesitamos reincorporar la arista $e$ en la descomposición. 

    \begin{itemize}
        \item Si $e$ no forma parte de un ciclo en $D'$, entonces se puede agregarla en cualesquiera de las dos subgráficas.
        \item Si $e$ crea un ciclo al unirse a una de las subgráficas, debemos colorcarla en otra subgráfica.
    \end{itemize}

    Entonces si $e$ no forma un ciclo en $D_1'$, implica que $e$ se coloca en $D_1$, quedando como, $A_1=A_1'\cup \{e\}$. De otro modo si $e$ forma un ciclo en $D_1'$, implica que $e$ se coloca en $D_2$, como, $A_2= A_2' \cup \{e\}$.

    De tal forma que dicha construcción de gráfica, las subdigráficas $D_1=(V,A_1)$ y $D_2 = (V,A_2)$ son acíclicas. Cumpliendose los terminos de la condición:
    \begin{itemize}
        \item $A_1\cap A_2=\varnothing$
        \item $A_1\cup A_2 = A$ porque $A'_!\cup A_2'=A'$
    \end{itemize}

    Por inducción, hemos demostrado que cualquier digráfica sin lazos admite una descomposición en dos subdigráficas acíclicas disjuntas por aristas. $\blacksquare$

  \item Un torneo es una digr\'afica en la que entre cualesquiera
    dos v\'ertices existe una \'unica flecha.   Demuestre que todo
    torneo con más de dos vértices es fuertemente conexo o puede
    transformarse en un torneo fuertemente conexo al reorientar
    exactamente una flecha.

    
    Para demostrar esto, seguiremos los siguientes pasos:
    

    Sabemos que un torneo es una gáfica dirigida donde para cualquier par de vértices distintos \( u \) y \( v \), existe exactamente un arco entre ellos, ya sea \( (u, v) \) o \( (v, u) \), y que cuando es fuertemente conexo, es que para cualquier par de vértices \( u \) y \( v \), existe un camino dirigido desde \( u \) hasta \( v \) y viceversa.

        Caso base: Si el torneo tiene 3 vértices, hay dos posibilidades:
            \begin{itemize}
                \item {Fuertemente conexo:} Ciclos \( a \to b \to c \to a \).
                \item {No fuertemente conexo:} Por ejemplo, \( a \to b \to c \) y \( a \to c \). Aquí, no hay camino de \( c \) a \( a \) o \( b \). Reorientando \( b \to c \) a \( c \to b \), el torneo se vuelve fuertemente conexo: \( a \to b \leftarrow c \to a \).
            \end{itemize}

    
      Paso inductivo: 
      
      Supongamos que el torneo tiene \( n \geq 3 \) vértices. Si no es fuertemente conexo, entonces su grafica subyacente puede descomponerse en componentes fuertemente conexas \( T_1, T_2, \dots, T_k \) ordenadas tal que todos los arcos entre \( T_i \) y \( T_j \) (para \( i < j \)) van de \( T_i \) a \( T_j \).
      
      Si \( k \geq 2 \), elegimos un arco \( (u, v) \) donde \( u \in T_k \) y \( v \in T_1 \), movemos este arco a \( (v, u) \). 
           
                Existe un camino de \( T_1 \) a \( T_k \) a través de la reorientación.
                Las componentes \( T_1, T_2, \dots, T_k \) se fusionan en una única componente fuertemente conexa.
            
       
    
      Por lo tanto, en cualquier torneo no fuertemente conexo, reorientar un arco específico entre dos componentes fuertemente conexas garantiza que el grafo resultante sea fuertemente conexo.
   
      

    
  \item Demuestre que una digr\'afica es fuertemente conexa si
    y s\'olo si contiene un camino cerrado generador.
    
 \underline{Respuesta: } 

($\Rightarrow$) Si es fuertemente conexa, entonces contiene un camino cerrado generador.


Sea $G$ una digráfica fuertemente conexa con vértices $V = \{v_1, v_2, \dots, v_n\}$. Por ser fuertemente conexa, existe un camino de $v_i$ a $v_j$ para cualquier par de vértices.

Construimos un camino cerrado que visita todos los vértices, comenzamos en $v_1$ y seguimos un camino a $v_2$. De $v_2$ seguimos a $v_3$, y así sucesivamente hasta $v_n$ y como $G$ es fuertemente conexa, existe un camino de $v_n$ de vuelta a $v_1$.
   
La concatenación de estos caminos forma un camino cerrado que visita todos los vértices.



($\Leftarrow$) Si contiene un camino cerrado generador, entonces es fuertemente conexa.


Supongamos que $G$ tiene un camino cerrado generador $C = (w_1 \to w_2 \to \dots \to w_k \to w_1)$ que pasa por todos los vértices.


    Para cualquier par de vértices $u$ y $v$:
    $u$ y $v$ aparecen en $C$, digamos $u = w_i$ y $v = w_j$.
    Si $i < j$, hay un camino de $u$ a $v$ siguiendo $C$.
    Si $i > j$, hay un camino de $u$ a $v$ siguiendo $C$ hasta $w_k$ y luego a $w_1$ hasta $w_j$.
    
    Por tanto, existe un camino entre cualquier par de vértices en ambas direcciones.



\end{enumerate}

\end{document}

